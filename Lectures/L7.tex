\documentclass{article}
\usepackage{amssymb}
\usepackage{amsmath}
\usepackage{centernot}
\usepackage{multicol}
\usepackage[pdftex]{graphicx}
\usepackage{listings}
\setlength{\parindent}{0in}

\begin{document}

\title{EE313\: Lecture 7}
\author{Joshua Dong}
\date{\today}
\maketitle

\begin{enumerate}
    \item{Midterm October 1 (chapter 1, 2)}
    \item{Homework 3 from 2.1}
\end{enumerate}

\subsection{Recap}
Last time proved LTI property: convolution on $h[n-k]$
\\$h[n-k] \rightarrow$ know if LTI.

\subsection{Convolutions}
New operator:
$x[n] * h[n] = \sum\limits_{k = -\infty}^\infty x[k] \cdot h[n-k]$
\\Note: convolution is commutative.
\begin{description}
    \item{Example:} \hfill \\
        An LTI system has impulse response
        $h[n] = \delta[n] + \delta[n-1]$.
        Find the response to the input $x_1[n] = \delta[n]$.
    \item{Impulse response} \hfill \\
        output $y[n] = h[n] = \delta[n] + \delta[n+1]$ by definition of
        impulse response.
        \\$y[n] = \sum\limits_{k = -\infty}^\infty x[k] \cdot h[n-k] =
        x[0] \cdot h[n-0]$.
        (See image, x[n] is zero everywhere but when $n=0$)
    \\ 
    \item{Example2:} \hfill \\
        Find the response to the input $x_2[n] = \delta[n-2]$.
    \\ 
    \item{Example3:} \hfill \\
        An LTI sytem has an impulse response
        $h[n] = \delta[0]+ \delta[1] + \delta[2]$
        \\Find output when 
        $x[n] = 0.5\delta[0]+ 2\delta[1]$.
    
\end{description}


\end{document}
