\documentclass{article}
\usepackage{amssymb}
\usepackage{amsmath}
\usepackage{centernot}
\usepackage{multicol}
\usepackage[pdftex]{graphicx}
\usepackage{listings}
\setlength{\parindent}{0in}

\begin{document}

\title{EE313\: Lecture 6}
\author{Joshua Dong}
\date{\today}
\maketitle

\subsection{LTI Systems}
Properties of systems:
\begin{enumerate}
    \item{Memory}
    \item{Invertability}
    \item{Causality}
    \item{Stability}
    \item{Time Invariance}
    \item{Linearity}
\end{enumerate}
e.g. $y[n] = nx[n]$
Recall: \\
$x_1[n] \xrightarrow{S} y_1[n] \rightarrow 
x_1[n-\delta] \xrightarrow{S} y_1[n-\delta]$.
\begin{multicols}{2}
$x_1[n] = \delta[n]
\\S[\delta[n]] = 0 \;\; \forall n.
\\x_1[n-5] = \delta[n-5]
\\S[\delta[n-5]] = 5\delta[n-5]$.
\end{multicols}
$\therefore$
Not time invariant.
\\\\e.g. $y(t) = x(2t)$
\\Not time invariant. Time is compressed by a factor of 2.
\\i.e Input $x_1 = u(t+2)-u(t-2)$.
\\$y_1(t) = u(2t+2) - u(2t-2)$.

\subsection{Recognizing Linear Systems}
Linearity: Rescaling + Superposition
\begin{description}
    \item[Any linear system has the zero-in-zero-out property.] \hfill \\
        $x[n] = 0 \; \forall n \in \mathbb{Z} \rightarrow g[n] = 0 \; \forall n \in \mathbb{Z} $.
\end{description}
e.g. $x[n-7] + 2 \cdot x[n-3] + 7$
\\This is not linear because it violates zero-in-zero-out property.
\\\\
\\Causality example:
\\$y(t) = x(t-7) +x(t-31) + 2t \cdot x(t-1) +
\int_{t-35}^{t-12}x(t)\log{(t^{2\cos{t}})}dt$
\\This system is causal.

\subsection{Chapter 2: Linear and Time Invariant Systems (LTI)}
\begin{quote}
    Very amazing but perhaps obvious is retrospect fact:
    \\For discrete time,
    \\I can use $\delta[n]$ to construct any discrete-time signal.
\end{quote}
For any signal x[n] I write it as a linear combination of shifted
$\delta[n]$'s:
\begin{quote}
    $y[n] = \sum\limits_{k = -\infty}^\infty $
\end{quote}
\subsubsection{Assuming Linearity}
Assume I have a system that is linear (but not yet Time Invariant).
\\Impulse Response: $x[n]=\delta[n] \rightarrow h_0[n]$. $h_0$ is the
impulse response.
\\Shifted Impulse Response: $x[n]=\delta[n-k] \rightarrow h_k[n]$.
\\If the system is Time Invariant, then
\\ $h_1[n] = h_0[n-1].
\\ h_k[n] = h_0[n-k].
\\\therefore y[n] = \sum\limits_{k = -\infty}^\infty x[k] \cdot h_0[n-k]$.


\end{document}
