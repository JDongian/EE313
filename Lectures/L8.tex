\documentclass{article}
\usepackage{amssymb}
\usepackage{amsmath}
\usepackage{centernot}
\usepackage{multicol}
\usepackage[pdftex]{graphicx}
\usepackage{listings}
\setlength{\parindent}{0in}

\begin{document}

\title{EE313\: Lecture 8}
\author{Joshua Dong}
\date{\today}
\maketitle

Midterm 1 no continuous time convolution

\subsection{Ch2}
Recall: $x[n] * h[n] = y[n] =
\sum\limits_{k = -\infty}^\infty x[k] \cdot h[n-k]$
\begin{description}
    \item{Continuous Time Convolution}
        $t \leftarrow n, \tau \leftarrow k, \int \leftarrow \sum
        \\\int_{\tau = -\infty}^\infty x[\tau] \cdot h[t-\tau]$
\end{description}

\subsection{LTI}
LTI system are compelety characterized by their impulse response.
Thus, you can find the output to any input if you know the impulse response.
(This is not true if the system is not LTI.)

\begin{description}
    \item{Example}
        Consider a discrete time system that has the following proprty:
        \\$\delta[n] \xrightarrow{S} Bob[n] = 
        \begin{cases} 
                0 & n\in\{0,1\} \\
                0 & otherwise
        \end{cases}$
    \item{Is the system LTI?}
        not enough information. If LTI:
    \item{1. Find the description equation y[n] = f(x[n])}
        LTI uniqueness property, start guessing.
        \\Try: $y[n] = x[n] + x[n+1]$
        \\Then $y[-5] = 0, y[-1] = 1, y[0] = 1, y[1] = 0$.
        \\
        \\Try: $y[n] = x[n] + x[n-1]$
        \\Then $y[-1] = 0, y[0] = 1, y[1] = 1, y[2] = 0$.
        \\This is the correct output.

    \item{2. Find the output when $x[n] = \delta[n+1] + \delta[n]$}

    \item{3. Is there another system that has the same output for that input?}
        No.

\end{description}


\end{document}
