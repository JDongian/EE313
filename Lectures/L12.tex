\documentclass{article}
\usepackage{amssymb}
\usepackage{amsmath}
\usepackage{centernot}
\usepackage{multicol}
\usepackage[pdftex]{graphicx}
\usepackage{listings}
\setlength{\parindent}{0in}

\begin{document}

\title{EE313\: Lecture 11}
\author{Joshua Dong}
\date{\today}
\maketitle

Fourier, fundamental period of $e^{j\frac{2\pi}{T}t}$ is $T$.
\\Similarly, fundamental period of $e^{j3 \cdot \frac{2\pi}{T}t}$ is
$\frac{T}{3}$.
\\
\\
Dirichlet conditions for some $x(t)$
\begin{description}
    \item{C1: Absolute integrability}
    \item{C2: Any finite interval has a finite number of maxima and minima}
        Examine: $sin(\frac{1}{n})$ and constructions of Cantor dust.
    \item{C3: }
\end{description}
All three conditions satisfied imply that
$x(t) = \sum{a_k \cdot e^{j\frac{2\pi}{T}t}}$.
\\How to find the Fourier series coefficients $a_k$?
\\Note that the Fourier series is the projection of a function onto the
frequency domain. Thus each $a_k$ represents a projection of $x(t)$ onto
the ferquency.
\\
\\Define the dot product between two continuous time functions as:
\\$<x_1(t), x_2(t)> = \int_0^\tau f(t) \cdot g^*(t) dt$.
\\We call this a Hilbert space.

\end{document}
